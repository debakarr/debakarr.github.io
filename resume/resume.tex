\documentclass{muratcan_cv}

\setname{Debakar Roy}{}
\setmobile{+91 8391887283}
\setmail{debakar.roy@outlook.com}
\setlinkedinaccount{https://www.linkedin.com/in/debakar-roy/} 
\setgithubaccount{https://github.com/debakarr} 
\setwebsite{https://debakarr.github.io/} 
\setthemecolor{MidnightBlue}

    
\begin{document}

%Create header
\headerview
\vspace{1ex} % white space
%
\section{Experience}
%
\datedexperience{Intel Corporation}{August 2019 - Present}
\explanation{Software Engineer}{Karnataka, IN}
\explanationdetail{
	\smallskip
	\coloredbullet\ %
	Developed and implemented automation solutions in multiple domains, including power management, sensor technology, artificial intelligence (computer vision), security, and type-C subsystems.
	    
	\smallskip 
	\coloredbullet\ %
	Utilized industry-standard DevOps tools to establish an efficient continuous integration and delivery (CI/CD) pipeline for delivering the framework and libraries.
	     
	\smallskip
	\coloredbullet\ %
	Established a remote execution environment that enabled testers to execute scenarios without the need to be physically present in front of the test setup, resulting in increased efficiency and flexibility.
	
	\smallskip
	\coloredbullet\ %
	Served as a core contributor and maintainer of the framework, responsible for overseeing the code review process to maintain high standards of quality throughout the code base.
	
	\smallskip
	\coloredbullet\ %
	Leverage Knowledge in Python, Poetry (Python package manager), Jenkins, Docker, Kubernetes, Terraform, VNC, Serial port communication.
	     
	\smallskip
}
%
\datedexperience{Intel Corporation}{July 2018 - August 2019} 
\explanation{Undergrade Intern Technical}{Karnataka, IN} 
\explanationdetail{ 
	\smallskip 
	\coloredbullet\ %
	RFKPI Automation: Python based in-house automation framework to automate RF performance and compliance to 3GPP for all operating modes (2G, 3G, and 4G) during development and production.
	     
	\smallskip
	\coloredbullet\ %
	Leverage knowledge in Git, Intel XMM modems, programmed in Python using PyCharm IDE, run on CMW and Anritsu Wideband Radio Communication Tester.
	     
	%  \smallskip
	\bigskip
} 
%
\section{Education} 
\datedexperience{Bachelor of Technology, Computer Science}{August 2019 - July 2019}
\explanation{Siliguri Institute of Technology, 9.12 DGPA}{West Bengal, IN} 
% Skills
\section{Skills}
%
\newcommand{\skillone}{\createskill{Programming Languages}{\textbf{\emph{Experienced:}} \ \  Python \ \ \textbf{\emph{Familiar:}} \ \  Rust \cpshalf Bash \cpshalf C++ \cpshalf JavaScript}}
%
\newcommand{\skilltwo}{\createskill{Software Development}{Programming Paradigms \cpshalf Design Patterns \cpshalf Agile methodologies \cpshalf Git \cpshalf REST \cpshalf CLI \cpshalf CI/CD}}
%
\newcommand{\skillthree}{\createskill{Frameworks \ \& \ Libraries}{FastAPI \cpshalf Flask \cpshalf Vue \cpshalf Pandas \cpshalf Selenium \cpshalf Playwright \cpshalf Poetry \cpshalf HUGO}}
%
\newcommand{\skillfour}{\createskill{DevOps}{Jenkins \cpshalf Kubernetes \cpshalf Docker \cpshalf CloudFoundry \cpshalf Rancher \cpshalf Terraform}}
%
\newcommand{\skillfive}{\createskill{Languages}{\textbf{\emph{Native:}} \ \  Hindi \ \ \textbf{\emph{Fluent:}} \ \ English \ \ \textbf{\emph{Beginner:}} \ \  Japanese \cpshalf Korean \cpshalf Mandarin }}
%
\createskills{\skillone, \skilltwo, \skillthree, \skillfour, \skillfive}
%
\section{Projects}
%
\datedexperience{kisskh-dl}{July 2022}
\explanation{Simple CDrama/KDrama downloader}{\href{https://github.com/debakarr/kisskh-dl}{\faGithub\ \footnotesize Source}}
\explanationdetail{
	\smallskip 
	\coloredbullet\ %
	Developed a python package with a command line interface to download CDrama/KDrama. 
	     
	     
	\smallskip
	\coloredbullet\ %
	Incorporate conversion of m3u8 stream into mp4/mkv.
	     
	\smallskip
	\coloredbullet\ %
	Implement the CI/CD to deploy a Python package in PyPI when new tag is applied to the repository.
	     
	\smallskip
	\coloredbullet\ %
	Utilized: Python, requests, click. 
	     
	\smallskip
}
%
\datedexperience{Fetch Random Technical Tweets}{December 2022}
\explanation{Fetch random technical tweets from pre-defined Twitter users}{\href{https://fetch-random-tech-tweet.vercel.app/}{\faGithub\ \footnotesize Source}} 
\explanationdetail{
	\smallskip
	\coloredbullet\ %
	Developed a Vue application and deployed it in Vercel. 
	    
	\smallskip 
	\coloredbullet\ %
	Integrated the app with a REST API backed up with json-server in Vercel.
	      
	\smallskip
	\coloredbullet\ %
	Developed a scheduled automated pipeline which updates the database used by REST API.  
	
	\smallskip
	\coloredbullet\ %
	Utilized: Vue, Python, Twitter API, json-server. 
	     
	\smallskip
}
%
\datedexperience{Flappy NEAT}{February 2018}
\explanation{Play Flappy Bird using NeuroEvolution of Augmenting Topologies}{\href{https://github.com/debakarr/Flappy-Bird-using-NeuroEvolution-of-Augmenting-Topologies/}{\faGithub\ \footnotesize Source}} 
\explanationdetail{
	\smallskip
	\coloredbullet\ %
	Implementation of NeuroEvolution of Augmenting Topologies algorithm and visualizing the result with the help of Flappy Bird game i.e. using Genetic Algorithm to evolve Neural Networks.
	    
	\smallskip 
	\coloredbullet\ %
	Utilized: Python, Numpy, pygame.
	
	\smallskip
	%
}
\datedexperience{Myanimelist Data Set Creator}{February 2018}
\explanation{Parse and store data of Myanimelist for Data Analytics}{\href{https://github.com/debakarr/myanimelist-data-set-creator/}{\faGithub\ \footnotesize Source}}
\explanationdetail{
	\smallskip
	\coloredbullet\ %
	Developed using Python to create MyAnimeList Anime and User data set using REST API. Planned on using it for a recommendation system.
	    
	\smallskip 
	\coloredbullet\ %
	Utilized: Python, Numpy, pygame.
	
	\smallskip
	%  \bigskip
}
%
\end{document}
